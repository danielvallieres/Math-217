\documentclass{amsart} 
\usepackage{amssymb,latexsym,amsmath,amscd,graphics,amsthm,verbatim,url}
%\usepackage[margin = 3 cm]{geometry}


\theoremstyle{plain}
%Create a problem environment to type up the questions.
\newtheorem{problem}{Problem}
      
%Create an environment Solution where the students will write their solutions.
\newenvironment{solution}{\paragraph{\emph{Solution 1}.}}{\hfill$\square$}


%Create an environment for the second try if necessary.
\newenvironment{sectry}{\paragraph{\emph{Solution 2}.}}{\hfill$\square$}


\begin{document} 

\title[Homework 3]{Homework 3}
\author{My name is:  WRITE YOUR NAME HERE}  %Replace WRITE YOUR NAME HERE by your actual full name
\date{\today} 
\maketitle 

\begin{problem}
Do problems 26 and 28 from your book on page 92.
\end{problem}
\begin{solution}
% Write your solution here.
\end{solution}

\begin{problem}
Represent $1000111_{2}$ in decimal notation.
\end{problem}
\begin{solution}
% Write your solution here.
\end{solution}

\begin{problem}
Another basis that is often used to represent positive integers is $16$ instead of $10$ (decimal) or $2$ (binary).  This representation is called the hexadecimal representation.  Since we need one character per position, one uses $A$ for $10$, $B$ for $11$, $C$ for $12$, $D$ for $13$, $E$ for $14$, and $F$ for $15$.  For instance
$$3CF_{16} = 975_{10}. $$ 
Go ahead and find the hexadecimal representation of $2619_{10}$.  (Hint:  Adapt the procedure we saw in class for base $2$ to base $16$.)  Practice some more on your own to write the binary and hexadecimal representation of positive integers.
\end{problem}
\begin{solution}
% Write you solution here.

\end{solution}

\begin{problem} 
Do problem 21 on page 107 of your book.
\end{problem}
\begin{solution}
% Write your solution here.

\end{solution}


\begin{problem}
Read the section \emph{Circuits for Computer Addition} in your book on page 97, 98, and 99.  Then draw a digital circuit that allows one to add two binary numbers of length four.  If interested, you can go to 
\begin{center}
\url{https://logic.ly/demo/samples} 
\end{center}
and construct digital circuits using this website.
\end{problem}
\begin{solution}
% Write your solution here.
\end{solution}

\begin{problem}
Let $Q(x,y)$ be the predicate ``If $x<y$ then $x^{2}<y^{2}$'' with domain for both $x$ and $y$ being $\mathbb{R}$.
\begin{enumerate}
\item Explain why $Q(x,y)$ is false if $x = -2$ and $y=1$. \label{first}
\item Give values different from those in part (\ref{first}) for which $Q(x,y)$ is false.
\item Explain why $Q(x,y)$ is true if $x=3$ and $y=8$. \label{third}
\item Give values different from those in part (\ref{third}) for which $Q(x,y)$ is true.
\end{enumerate}
\end{problem}

\begin{solution}
% Write your solution here.

\end{solution}

\begin{problem}
Consider the quantified statement:
\begin{equation*}
\begin{aligned}
&x: \mathbb{R}\\
&\forall x, x \ge 1/x.
\end{aligned}
\end{equation*}
Is this statement true or false?  If false, find a counterexample.
\end{problem}

\begin{solution}
% Write your solution here.

\end{solution}


\begin{problem}
Consider the quantified statement:
\begin{equation*}
\begin{aligned}
&x,y : \mathbb{R}\\
&\forall x,\forall y, (x + y)^2 = x^2 + y^2.
\end{aligned}
\end{equation*}
Is this statement true or false?  If false, find a counterexample.
\end{problem}

\begin{solution}
% Write your solution here.

\end{solution}

\begin{problem}
Consider the quantified statement:
\begin{equation*}
\begin{aligned}
&x,y : \mathbb{R}\\
&\exists x,\exists y, (x + y)^2 = x^2 + y^2.
\end{aligned}
\end{equation*}
Is this statement true or false?  If true, give an example.
\end{problem}

\begin{solution}
% Write your solution here.

\end{solution}

\begin{problem}
Consider the following quantified statement:
\begin{equation*}
\begin{aligned}
&x : \mathbb{R}\\
&\exists x, x^{2} = 2.
\end{aligned}
\end{equation*}
Which of the following are equivalent ways of expressing this statement?
\begin{enumerate}
\item The square of each real number is $2$.
\item Some real numbers have square $2$.
\item The number $x$ has square $2$, for some real number $x$.
\item If $x$ is a real number, then $x^{2} = 2$.
\item Some real number has square $2$.
\item There is at least one real number whose square is $2$.
\end{enumerate}
\end{problem}

\begin{solution}
% Write your solution here.

\end{solution}


\begin{problem}
Let $\mathbb{R}$ be the domain of the predicate variable $x$.  Which of the following are true and which are false?  Give counterexamples for the statements that are false.
\begin{enumerate}
\item $x > 2 \Rightarrow x > 1$
\item $x > 2 \Rightarrow x^{2} > 4$
\item $x^{2} > 4 \Rightarrow x > 2$
\item $x^{2}>4 \Leftrightarrow |x| > 2$
\end{enumerate}
\end{problem}

\begin{solution}
% Write your solution here.

\end{solution}



\end{document}



