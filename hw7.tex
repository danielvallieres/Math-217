\documentclass{amsart} 
\usepackage{amssymb,latexsym,amsmath,amscd,graphics,amsthm,verbatim,url}
%\usepackage[margin = 3 cm]{geometry}


\theoremstyle{plain}
%Create a problem environment to type up the questions.
\newtheorem{problem}{Problem}
      
%Create an environment Solution where the students will write their solutions.
\newenvironment{solution}{\paragraph{\emph{Solution 1}.}}{\hfill$\square$}


%Create an environment for the second try if necessary.
\newenvironment{sectry}{\paragraph{\emph{Solution 2}.}}{\hfill$\square$}


\begin{document} 

\title[Homework 7]{Homework 7}
\author{My name is:  WRITE YOUR NAME HERE}  %Replace WRITE YOUR NAME HERE by your actual full name
\date{\today} 
\maketitle 

\begin{problem}
Do problems 1 and 2 in your book on page 435.
\end{problem}
\begin{solution}
% Write your solution here.
\end{solution}

\begin{problem}
Define the function $S:\mathbb{Z}_{>0} \rightarrow \mathbb{Z}_{>0}$ via 
$$n \mapsto S(n) = \text{ the sum of the positive divisors of } n.$$
Find the following:
\begin{enumerate}
\item $S(1)$
\item $S(15)$
\item $S(17)$
\item $S(5)$
\item $S(18)$
\item $S(21)$
\end{enumerate}
\end{problem}
\begin{solution}
% Write your solution here.
\end{solution}

\begin{problem}
Define the function $T:\mathbb{Z}_{>0} \rightarrow \mathcal{P}(\mathbb{Z}_{>0})$ via
$$m \mapsto T(n) = \text{ the set of positive divisors of }n. $$
Find the following:
\begin{enumerate}
\item $T(1)$
\item $T(15)$
\item $T(17)$
\item $T(5)$
\item $T(18)$
\item $T(21)$
\end{enumerate}
\end{problem}
\begin{solution}
% Write your solution here.
\end{solution}


\begin{problem}
Consider the function $f:\mathbb{Z} \times \mathbb{Z} \rightarrow \mathbb{Z} \times \mathbb{Z}$ defined via
$$(a,b) \mapsto f(a,b) = (2a+1,3b-2). $$
Find the following:
\begin{enumerate}
\item $f(4,4)$
\item $f(2,1)$
\item $f(3,2)$
\item $f(1,5)$
\end{enumerate}
\end{problem}
\begin{solution}
% Write your solution here.
\end{solution}



\begin{problem}
Recall the \emph{floor} and the \emph{ceiling} functions.  The floor function is a function $\lfloor \, \,\,\rfloor:\mathbb{R} \rightarrow \mathbb{Z}$ defined via
$$x \mapsto \lfloor x \rfloor = \text{ the unique integer } n \text{ satisfying } n \le x < n+1,$$
and the ceiling function $\lceil \, \, \, \rceil:\mathbb{R} \rightarrow \mathbb{Z}$ is defined via
$$x \mapsto \lceil x \rceil = \text{ the unique integer } n \text{ satisfying } n-1 < x \le n.$$

Define now two new functions $H, K:\mathbb{R} \rightarrow \mathbb{R}$ via
$$x \mapsto H(x) = \lfloor x \rfloor + 1 $$
and
$$x \mapsto K(x) = \lceil x \rceil. $$
Does $H = K$?  Explain.
\end{problem}
\begin{solution}
% Write your solution here.
\end{solution}

\begin{problem}
Let $f:A \rightarrow B$ be a function and let $A_{1}, A_{2}$ be subsets of $A$ satisfying $A_{1} \subseteq A_{2}$.  Do we always have
$$f(A_{1}) \subseteq f(A_{2})? $$
If yes, prove it.  Otherwise, give a counterexample.
\end{problem}
\begin{solution}
% Write your solution here.
\end{solution}

\begin{problem}
Let $f:A \rightarrow B$ be a function and let $A_{1}, A_{2}$ be subsets of $A$.  Do we always have
$$f(A_{1} \cap A_{2}) = f(A_{1}) \cap f(A_{2})? $$
If yes, prove it.  Otherwise, give a counterexample.
\end{problem}
\begin{solution}
% Write your solution here.
\end{solution}

\begin{problem}
Let $f:A \rightarrow B$ be a function and let $B_{1}, B_{2}$ be subsets of $B$.  Do we always have
$$f^{-1}(B_{1} \cap B_{2}) = f^{-1}(B_{1}) \cap f^{-1}(B_{2})? $$
If yes, prove it.  Otherwise, give a counterexample.
\end{problem}
\begin{solution}
% Write your solution here.
\end{solution}


\end{document}



