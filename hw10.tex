\documentclass{amsart} 
\usepackage{amssymb,latexsym,amsmath,amscd,graphics,amsthm,verbatim,url}
%\usepackage[margin = 3 cm]{geometry}


\theoremstyle{plain}
%Create a problem environment to type up the questions.
\newtheorem{problem}{Problem}
      
%Create an environment Solution where the students will write their solutions.
\newenvironment{solution}{\paragraph{\emph{Solution 1}.}}{\hfill$\square$}


%Create an environment for the second try if necessary.
\newenvironment{sectry}{\paragraph{\emph{Solution 2}.}}{\hfill$\square$}


\begin{document} 

\title[Homework 10]{Homework 10}
\author{My name is:  WRITE YOUR NAME HERE}  %Replace WRITE YOUR NAME HERE by your actual full name
\date{\today} 
\maketitle 

\begin{problem}
Let $k,n \in \mathbb{Z}$.  If $n = 4k + 1$, then is it true that $8 \mid n^{2} - 1$?  Explain your answer. 
\end{problem}
\begin{solution}
% Write your solution here.
\end{solution}

\begin{problem}
Prove carefully that the sum of any three consecutive integers is divisible by $3$.
\end{problem}
\begin{solution}
% Write your solution here.
\end{solution}

\begin{problem}
Prove carefully that the product of two even integers is divisible by $4$.  (Recall that an integer is even if it's divisible by $2$.)
\end{problem}
\begin{solution}
% Write your solution here.
\end{solution}

\begin{problem}
Prove carefully that if $a,b \in \mathbb{Z}$, and $a \neq 0$, the divisibility $a \mid b$ implies the divisibility $a^{2} \mid b^{2}$. 
\end{problem}
\begin{solution}
% Write your solution here.
\end{solution}


\begin{problem}
Let $a,b \in \mathbb{Z}$ with $a \neq 0$.  Prove carefully that if $a \mid b$ and $a \mid c$, then $a \mid (b+c)$.
\end{problem}
\begin{solution}
% Write your solution here.
\end{solution}


\begin{problem}
Define a relation $R$ on $\mathbb{R}^{2}$ as follows:  (a,b) R (c,d) if either $a < c$ or both $a=c$ and $b \le d$.  Is $R$ a partial order relation?  Prove or give a counterexample.
\end{problem}
\begin{solution}
% Write your solution here.
\end{solution}

\begin{problem}
Define a relation $R$ on $\mathbb{Z}$ as follows:  $m R n$ if $m + n$ is even.  Is $R$ a partial order relation?  Prove or give a counterexample.
\end{problem}
\begin{solution}
% Write your solution here.
\end{solution}



\end{document}



