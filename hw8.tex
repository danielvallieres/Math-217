\documentclass{amsart} 
\usepackage{amssymb,latexsym,amsmath,amscd,graphics,amsthm,verbatim,url}
%\usepackage[margin = 3 cm]{geometry}


\theoremstyle{plain}
%Create a problem environment to type up the questions.
\newtheorem{problem}{Problem}
      
%Create an environment Solution where the students will write their solutions.
\newenvironment{solution}{\paragraph{\emph{Solution 1}.}}{\hfill$\square$}


%Create an environment for the second try if necessary.
\newenvironment{sectry}{\paragraph{\emph{Solution 2}.}}{\hfill$\square$}


\begin{document} 

\title[Homework 8]{Homework 8}
\author{My name is:  WRITE YOUR NAME HERE}  %Replace WRITE YOUR NAME HERE by your actual full name
\date{\today} 
\maketitle 

\begin{problem}
Consider the function $f:\mathbb{Z} \rightarrow \mathbb{Z}$ defined via $n \mapsto f(n) = 2 - 3n$.
\begin{enumerate}
\item Is $f$ injective?  Explain carefully.
\item Is $f$ surjective?  Explain carefully.
\item Is $f$ bijective?  Explain carefully.
\end{enumerate}
\end{problem}
\begin{solution}
% Write your solution here.
\end{solution}

\begin{problem}
Consider the function $f:\mathbb{R} \smallsetminus \{ 0\} \rightarrow \mathbb{R}$ given by 
$$x \mapsto f(x) = \frac{x+1}{x}. $$
\begin{enumerate}
\item Is $f$ injective?  Explain carefully.
\item Is $f$ surjective?  Explain carefully.
\item Is $f$ bijective?  Explain carefully.
\end{enumerate}
\end{problem}
\begin{solution}
% Write your solution here.
\end{solution}

\begin{problem}
Recall from Problem 5 on Homework 7 the floor function $\lfloor \, \, \rfloor:\mathbb{R} \rightarrow \mathbb{Z}$.
\begin{enumerate}
\item Is the floor function injective?  Explain carefully.
\item Is the floor function surjective?  Explain carefully.
\item Is the floor function bijective?  Explain carefully.
\end{enumerate}
\end{problem}
\begin{solution}
% Write your solution here.
\end{solution}


\begin{problem}
Let $X = \{ a,b,c\}$, and consider the function $f:\mathcal{P}(X) \rightarrow \mathbb{Z}$ given by
$$A \mapsto f(A) = \text{the number of elements in }A. $$
\begin{enumerate}
\item Is $f$ injective?  Explain carefully.
\item Is $f$ surjective?  Explain carefully.
\item Is $f$ bijective?  Explain carefully.
\end{enumerate}
\end{problem}
\begin{solution}
% Write your solution here.
\end{solution}



\begin{problem}
Let $f:\mathbb{R}^{2} \rightarrow \mathbb{R}^{2}$ be the function defined via
$$(x,y) \mapsto f(x,y) = (2y,-x). $$
\begin{enumerate}
\item Is $f$ injective?  Explain carefully.
\item Is $f$ surjective?  Explain carefully.
\item Is $f$ bijective?  Explain carefully.
\end{enumerate}
\end{problem}
\begin{solution}
% Write your solution here.
\end{solution}

\begin{problem}
Consider the function $f:\mathbb{R} \rightarrow \mathbb{R}$ given by $x \mapsto f(x) = x^{3}$ and the function $g:\mathbb{R} \rightarrow \mathbb{R}$ given by $x \mapsto g(x) = x-1$.  Calculate
\begin{enumerate}
\item $g \circ f (2)$
\item $f \circ g (2)$
\end{enumerate}
\end{problem}
\begin{solution}
% Write your solution here.
\end{solution}

\begin{problem}
Let $f:A \rightarrow B$ be a function.  Define a relation $\sim$ on $A$, via
$$a_{1} \sim a_{2} \text{ if } f(a_{1}) = f(a_{2}). $$
\begin{enumerate}
\item Show carefully that $\sim$ is an equivalence relation on $A$.
\item Consider now the particular function $f:\mathbb{R} \rightarrow \mathbb{R}$ given by $x \mapsto f(x) = x^{2}$.  List the elements in $[1], [-2],$ and $[0]$.
\end{enumerate}
\end{problem}
\begin{solution}
% Write your solution here.
\end{solution}

\begin{problem}
Let $f:X \rightarrow Y$ and $g:Y \rightarrow Z$ be two functions such that $g \circ f$ is injective.  Does that imply that $g$ is injective?  If yes, prove this, otherwise, give a counterexample.
\end{problem}
\begin{solution}
% Write your solution here.
\end{solution}


\end{document}



