\documentclass{amsart} 
\usepackage{amssymb,latexsym,amsmath,amscd,graphics,amsthm,verbatim,url}
%\usepackage[margin = 3 cm]{geometry}


\theoremstyle{plain}
%Create a problem environment to type up the questions.
\newtheorem{problem}{Problem}
      
%Create an environment Solution where the students will write their solutions.
\newenvironment{solution}{\paragraph{\emph{Solution 1}.}}{\hfill$\square$}


%Create an environment for the second try if necessary.
\newenvironment{sectry}{\paragraph{\emph{Solution 2}.}}{\hfill$\square$}


\begin{document} 

\title[Homework 5]{Homework 5}
\author{My name is:  WRITE YOUR NAME HERE}  %Replace WRITE YOUR NAME HERE by your actual full name
\date{\today} 
\maketitle 

\begin{problem}
Union and intersection of sets can be extended to more than two sets.  Read first the definition on page 383 of your book if necessary before attempting the following problem.  For each $i \in \mathbb{Z}_{\ge 0}$, let
$$D_{i} = [-i,i] \subseteq \mathbb{R}. $$
For each of the followings, tell me what set you get.
\begin{enumerate}
\item $\bigcup\limits_{i=0}^{4} D_{i}$
\item $\bigcap\limits_{i=0}^{4} D_{i}$
\item $\bigcup\limits_{i=0}^{n} D_{i}$
\item $\bigcup\limits_{i=0}^{n} D_{i}$
\item $\bigcup\limits_{i=0}^{\infty}D_{i}$
\item $\bigcap\limits_{i=0}^{\infty}D_{i}$
\end{enumerate} 
\end{problem}
\begin{solution}
% Write your solution here.
\end{solution}

\begin{problem}
For each $i \in \mathbb{Z}_{> 0}$, let
$$D_{i} = \big[-\frac{1}{i},\frac{1}{i}\big] \subseteq \mathbb{R}. $$
For each of the followings, tell me what set you get.
\begin{enumerate}
\item $\bigcup\limits_{i=0}^{4} D_{i}$
\item $\bigcap\limits_{i=0}^{4} D_{i}$
\item $\bigcup\limits_{i=0}^{n} D_{i}$
\item $\bigcup\limits_{i=0}^{n} D_{i}$
\item $\bigcup\limits_{i=0}^{\infty}D_{i}$
\item $\bigcap\limits_{i=0}^{\infty}D_{i}$
\end{enumerate}
\end{problem}
\begin{solution}
% Write your solution here.
\end{solution}

\begin{problem}
Suppose $A = \{1,2 \}$ and $B = \{ 2,3\}$.  Find each of the following:
\begin{enumerate}
\item $\mathcal{P}(A \cap B)$
\item $\mathcal{P}(A)$
\item $\mathcal{P}(A \cup B)$
\item $\mathcal{P}(A \times B)$
\end{enumerate}
\end{problem}
\begin{solution}
% Write your solution here.
\end{solution}


\begin{problem}
Let $A = \{ a,b\}$, $B  =\{1,2\}$ and $C = \{ 2,3\}$.  Find each of the following sets.
\begin{enumerate}
\item $A \times (B \cup C)$
\item $(A \times B) \cup (A \times C)$
\item $A \times (B \cap C)$
\item $(A \times B) \cap (A \times C)$
\end{enumerate}
\end{problem}
\begin{solution}
% Write your solution here.
\end{solution}

\begin{problem}
Find $\mathcal{P}(\mathcal{P}(\mathcal{P}(\varnothing)))$
\end{problem}
\begin{solution}
% Write your solution here.
\end{solution}

\begin{problem}
Provide a proof of the following equality of sets:
$$(A \cap B) \cup (A \cap B^{c}) = A.$$
\end{problem}
\begin{solution}
% Write your solution here.
\end{solution}

\begin{problem}
Provide a proof of the following equality of sets:
$$A \cup (A \cap B) = A.$$
\end{problem}
\begin{solution}
% Write your solution here.
\end{solution}

\begin{problem}
Consider the following claim:  For all sets $A,B,C$, one has
$$(A \cup B) \cap C = A \cup (B \cap C). $$
If it is true, prove this identity.  If it is false, find a counterexample, in other words find three sets $A,B,C$ for which the identity is false.
\end{problem}
\begin{solution}
% Write your solution here.
\end{solution}


\end{document}



