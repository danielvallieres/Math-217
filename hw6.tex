\documentclass{amsart} 
\usepackage{amssymb,latexsym,amsmath,amscd,graphics,amsthm,verbatim,url}
%\usepackage[margin = 3 cm]{geometry}


\theoremstyle{plain}
%Create a problem environment to type up the questions.
\newtheorem{problem}{Problem}
      
%Create an environment Solution where the students will write their solutions.
\newenvironment{solution}{\paragraph{\emph{Solution 1}.}}{\hfill$\square$}


%Create an environment for the second try if necessary.
\newenvironment{sectry}{\paragraph{\emph{Solution 2}.}}{\hfill$\square$}


\begin{document} 

\title[Homework 6]{Homework 6}
\author{My name is:  WRITE YOUR NAME HERE}  %Replace WRITE YOUR NAME HERE by your actual full name
\date{\today} 
\maketitle 

\begin{problem}
Let $A = \{a,b,c \}$ and $B = \{1,2,3,4 \}$.  Then
$$R_{1} = \{(a,2),(a,3),(b,1),(b,3),(c,4) \} $$
is a relation from $A$ to $B$, while
$$R_{2} = \{(1,b),(1,c),(2,a),(2,b),(3,c),(4,a),(4,c) \} $$
is a relation from $B$ to $A$.  A relation $R$ is defined on $A$ by $x\, R \,y$ if there exists $z \in B$ such that $x \, R_{1} \, z$ and $z \, R_{2} \, y $.  Express $R$ by listing its elements.
\end{problem}
\begin{solution}
% Write your solution here.
\end{solution}

\begin{problem}
Let $H = \{2^{m}\, | \, m \in \mathbb{Z} \}$.  A relation $R$ is defined on the set $\mathbb{Q}_{>0}$ of positive rational numbers by $a \, R \, b$ if $a/b \in H$.
\begin{enumerate}
\item Show that $R$ is an equivalence relation.
\item Describe the elements in the equivalence class $[3]$.
\end{enumerate}
\end{problem}
\begin{solution}
% Write your solution here.
\end{solution}

\begin{problem}
Let $R$ be an equivalence relation on $A = \{a,b,c,d,e,f,g \}$ such that $a \, R \, c$, $c \, R \, d$, $d \, R \, g$ and $b \, R \, f$.  If there are three distinct equivalence classes resulting from $R$, then determine these equivalence classes and determine all elements of $R$.
\end{problem}
\begin{solution}
% Write your solution here.
\end{solution}


\begin{problem}
Consider $\mathbb{N} = \{1,2,\ldots \}$.  Define a relation $\sim$ on $\mathbb{N}^{2}$ via $(i,j) \sim (k,l)$ if $i+l = j + k$.   
\begin{enumerate}
\item Show carefully that $\sim$ is an equivalence relation on $\mathbb{N}^{2}$.  That is explain why the three defining properties of an equivalence relation are satisfied.
\item Enumerate five elements in $[(1,2)]$.
\item Enumerate five elements in $[(1,1)]$.
\end{enumerate}
\end{problem}
\begin{solution}
% Write your solution here.
\end{solution}

\begin{problem}
Consider the set $\Omega = \mathbb{Z} \times (\mathbb{Z} \smallsetminus \{ 0\})$ and define a relation on $\Omega$ via $(a,b) \sim (c,d)$ if $ad - bc = 0$.  
\begin{enumerate}
\item Show carefully that $\sim$ is an equivalence relation on $\Omega$.  That is explain why the three defining properties of an equivalence relation are satisfied.
\item Enumerate five elements in $[(1,2)]$.
\item Enumerate five elements in $[(1,1)]$.
\end{enumerate}
\end{problem}
\begin{solution}
% Write your solution here.
\end{solution}

\begin{problem}
Consider the set $\mathbb{Z}$ and the three sets
$$A_{0} = \{3k \, : \, k \in \mathbb{Z} \}, A_{1} = \{3k+1 \, : \, k \in \mathbb{Z} \}, \text{ and }  A_{2} = \{3k+2 \, : \, k \in \mathbb{Z} \}.$$
Do the sets $A_{0}, A_{1}, A_{2}$ form a partition of $\mathbb{Z}$??  Explain carefully.
\end{problem}
\begin{solution}
% Write your solution here.
\end{solution}

\begin{problem}
Let $X = \{ -1,0,1\}$ and $A = \mathcal{P}(X)$.  Define a relation $R$ on $A$ via $S \, R \, T$ if the sum of the elements in $S$ equals the sum of the elements in $T$.
\begin{enumerate}
\item Show carefully that $R$ is an equivalence relation on $A$.
\item Find the distinct equivalence classes.
\end{enumerate}
\end{problem}
\begin{solution}
% Write your solution here.
\end{solution}

\begin{problem}
Consider the set $A$ of all strings of length $4$ in $a$'s and $b$'s.  Define a relation $R$ on $A$ via $s \, R \, t$ if $s$ has the same first two characters as $t$.
\begin{enumerate}
\item Show carefully that $R$ is an equivalence relation on $A$.
\item Find the distinct equivalence classes.
\end{enumerate}
\end{problem}
\begin{solution}
% Write your solution here.
\end{solution}


\end{document}



